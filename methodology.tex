\section{Methods}

The starting configurations for EG and ABEN where proposed and optimized using XTB \cite{Bannwarth2021} with the force field GFN2-xTB \cite{Bannwarth2019}. The staring configuration for DBHET was proposed using crystallographic data \cite{Daubeny1954}, then reoptimized using $r^2\textrm{SCAN-3c}\textrm{def2/SVP}$. The final point energies where obtained all obtained using \chemomega-B97X-D4/def2-SVP and def2-TZVPP on just the oxygen, nitrogen, and  zinc atoms.
The optimized structures where then merged onto the same Cartesian coordinates and then re-optimized in $r^2\textrm{SCAN-3c}\textrm{def2/SVP}$. Then, a multidimensional relaxed surface scan was performed keeping constrained, and varying the bond distances in 20 steps to obtain the products of the reactions depicted in . The optimized product and the starting geometry of the relaxed surface scan where the input for an Energy-Weighted Climbing image Nudged Elastic Band (EW-CI-NEB) \cite{Asgeirsson2021} algorithm to find the path of minimum energy connecting both ends, followed by a P-RFO optimization to find a TS performed onto the climbing image (CI).

In order to determine covalent and non-covalent interactions regions between the molecules, an analysis based on Hirshfeld partition of molecular density (IGMH) algorithm within multiwfn was performed \cite{Lu2021} onto the optimized reactant. To minimize computation time, aIGM was performed with the MEP obtained by NEB-TS.

With JANPA \cite{Nikolaienko2014}, CLPOs and bond orders for the reactants, products and transition states where obtained.