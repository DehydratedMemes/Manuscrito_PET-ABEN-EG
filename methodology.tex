\section{Methods}

The starting configurations for EG and ABEN where proposed and optimized using XTB \cite{Bannwarth2021} with the force field GFN2-xTB \cite{Bannwarth2019}. The staring configuration for DBHET was proposed using crystallographic data \cite{Daubeny1954}. 

The software ORCA (5.0.1) \cite{Neese2020} was used to perform the geometrical optimization of the species involved in the depolymerization; PET dimer (DBHET), ABEN and EG in KS-DFT using a meta-NGA functional, MN15-L \cite{Yu2016} with the def2-TZVP BS and employing the option to use automatic auxiliary basis set generation \cite{Stoychev2017}.

The optimized structures where then merged onto the same Cartesian coordinates and then re-optimized in MN15-L/def2-SVP Then, a multidimensional relaxed surface scan was performed keeping constrained, and varying the bond distances in 5 steps to obtain the products of this reaction. The optimized product and the starting geometry of the relaxed surface scan where the input for an Energy-Weighted Nudged Elastic Band (EW-NEB)\cite{Asgeirsson2021} algorithm to find the path of minimum energy connecting both ends, followed by a P-RFO optimization to find a TS was performed onto the image with the highest energy alongside the MEP.

In order to determine interactions regions between the molecules, an IRI algorithm within multiwfn was utilized \cite{Lu2021} with the wavefunctions generated in the previous step. 

Using JANPA \cite{Nikolaienko2014}, CLPOs and the bond orders for the reactants, products and transition states where obtained from the wavefunction generated by ORCA. 